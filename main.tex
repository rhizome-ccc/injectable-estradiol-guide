\documentclass[twoside,a5paper]{article}

\usepackage{booktabs}
\usepackage{graphicx}

\begin{document}

\newcommand{\qrwidth}{6em}

\begin{titlepage}
  \null\vspace{2em}
  \setcounter{page}{0}
  \begin{center}
    \includegraphics[width=8em]{ouroboros} \\
    \vspace{3em}
    {\LARGE Injectable Estradiol Guide \\}
    \vspace{2em}
    {\large CCC \\}
    \vspace{0.6em}
    {\large v1.0.0 \\}
    \vspace{\stretch{2}}
  \end{center}
\end{titlepage}

\setcounter{page}{1}
\pagenumbering{roman}

\begin{center}
  \null\vfill
  \includegraphics[width=\qrwidth]{license-qr} \\
  \vspace{1.2em}
  \parbox{0.8\textwidth}{
    \small
    Copyright \textcopyright\ 2024 Cybernetic Communist Collective,
    published under Creative Commons Attribution-NonCommercial 4.0
    International.
  }
  \vfill
\end{center}

\newpage
\tableofcontents
\listoftables
\listoffigures

\newpage
\pagenumbering{arabic}

\noindent The following information is provided in the hope that it
will be useful, and effort has of course been put in to ensure its
accuracy.  However, the writers are not medical professionals,
researchers or experts in transgender health, so this should not be
regarded as a canonical source of information.  Individuals should do
their own research if they can and ideally consult a medical
professional if this is possible -- though of course it often is not,
hence why we are doing this in the first place.

With that in mind, if any information in this booklet is found to be
inaccurate or outdated, or if you have some additional information
you think should be included, please let us know so that we can update
it and inform others.  Solidarity!

\section{About Estradiol Monotherapy}

Hormone therapy for transfeminine people generally has two aims: to
suppress the body's production of testosterone and to increase
oestrogen levels.  This is commonly achieved with two separate
medications: an anti-androgen to suppress testosterone and some form
of oestrogen to increase oestrogen levels.  However, administration of
just oestrogen can achieve both of these as estradiol suppresses
testosterone production when at suitably high doses -- this is called
estradiol monotherapy.  As we are able to supply estradiol, but not
any anti-androgens, this is the route we recommend for those seeking
feminising effects.  It also avoids the additional side effects and
separate risk profile of an anti-androgen medication.

In order for monotherapy to be effective, a higher dose of estradiol
is required than is typically administered for combination therapy
(where an anti-androgen is also taken).  This may increase certain
certain risks, such as greater risk of blood clots.  For this reason
we recommend targeting an estradiol level of around 300 pg/mL (1100
pmol/L), which is high enough to reduce testosterone by more than 90\%
but is not massively higher than the upper end of the reference range
in cisgender women of 50-250 pg/mL.

\section{Recommended Starting Dosage}

Dosage depends on the target serum estradiol level, which is different
depending on whether an anti-androgen is being taken along with the
oestrogen.  For a target level of 300 pg/mL, appropriate for estradiol
monotherapy, a recommended starting dose is 5.5 mg of estradiol
enanthate (EEn), which is \textbf{0.11 mL of the provided 50 mg/mL
  oil, every 7 days}.  If an anti-androgen is also being taken along
with the estradiol, this can be reduced to \textbf{0.07 ml every 7
  days}, with a lower target level of around 200 pg/mL.

To determine an approximate dose for a different target level to those
above or for more or less frequent injections, \textit{Transfeminine
  Science} has an extremely useful tool which simulates serum
estradiol levels for different doses and intervals of injection for
different estradiol esters -- see figure~\ref{fig:simulator-qr}.

\begin{figure}
  \centering
  \includegraphics[width=\qrwidth]{simulator-qr}
  \caption{\small Transfeminine Science's injectable estradiol simulator}
  \label{fig:simulator-qr}
\end{figure}

It is important to note that these doses are based off averages and
will not be accurate for everyone.  The only definite way to get the
correct dose is to start at an average dose and make incremental
adjustments based off of the results seen in blood tests until the
target level is reached.  This process is called \textit{dose
  titration} and is recommended for anyone able to access blood tests.

\section{Risks \& Contraindications}

One risk to be aware of is the increase in blood clotting factors
associated with higher oestrogen levels.  This effect is not very
large, though may be significant for people that have heart disease,
other heart conditions or a history of these in their biological
family.  Oestrogens can also put additional strain on the liver,
though notably this may be significantly reduced with injections in
comparison to other routes of administration.  People with a very high
alcohol intake and those with liver problems or a history of them in
their biological family should do further research on this area and
weigh the risks.

Finally, there is the significant increase in breast cancer risk due
to breast tissue growth.  This is no higher with synthetic estradiol
than it is for breast tissue developed due to endogenously produced
oestrogens but should still be considered, especially for those with a
history of breast cancer in their biological family.  It is
recommended that people taking oestrogen sign up for breast cancer
screening where possible due to this increase in risk.

\section{Blood Tests \& Monitoring}

Where possible, people on hormone therapy should have blood tests
every three months after starting treatment, then every six months for
the following two years.  After this, testing only needs to be done
once a year.  Tests should ideally include serum estradiol, serum
testosterone, sex-hormone-binding globulin (SHBG), prolactin and
dihydrotestosterone (DHT) as well as liver function (LF) and a full
blood count (FBC).  Of these, estradiol and testosterone are the most
essential to track, though testing liver function is important for
people with an increased risk of developing liver problems, including
those with a high alcohol intake.  Bloods should ideally be taken
shortly before your next dose.  Table~\ref{tab:blood-ranges} lists
reference ranges for these tests (where applicable).

\begin{table}
  \centering
  \begin{tabular}{ll}
    \toprule
    Test & Reference range \\
    \midrule
    Estradiol & 300 pg/mL (monotherapy) \\
    & 200 pg/mL (with anti-androgen) \\
    Testosterone & $< 50$ ng/dL \\
    SHBG & $40 - 120$ nmol/L \\
    Prolactin & $< 1000$ mU/L \\
    DHT & $< 10$ ng/dL \\
    \bottomrule
  \end{tabular}
  \caption{\small Reference ranges for blood tests}
  \label{tab:blood-ranges}
\end{table}

Blood tests can be difficult to access, though for people with access
to an NHS GP, they will often do the tests as a harm reduction measure
if informed that you are self-medicating.  Otherwise, hormone level,
LF and FBC tests are available for free by appointment with CliniQ in
their clinic in South London (see figure~\ref{fig:cliniq-qr}), though
of course this requires traveling to London.  There are also DIY kits
available from many different sources online, though these are
generally very expensive and can be unreliable.

\begin{figure}
  \centering
  \includegraphics[width=\qrwidth]{cliniq-qr}
  \caption{\small CliniQ's page for their South London clinic}
  \label{fig:cliniq-qr}
\end{figure}

Hormone therapy without doing any blood work is much higher risk,
especially for those who already have increased baseline risk of liver
problems and heart disease.  People unable to access any blood tests
should do their own research and weigh the risks to come to an
informed decision.  Of any treatment route, estradiol monotherapy is
likely to be the lowest risk if no monitoring is possible as it avoids
the additional side effects of an anti-androgen, which can often be
worse than the risks from estradiol.  However, this may be different
for cases where blood clotting factors are the most significant
concern.

\section{Injecting Safely}

The EEn oil we provide should be administered as an intramuscular (IM)
injection.  This means the medication goes into muscle tissue, as
opposed to into fat, a blood vessel or under the skin.  IM injections
are extremely low risk as long as the appropriate steps are taken to
mitigate infection risk and a suitable injection site is used.

The two recommended injection sites are in the vastus lateralis muscle
in the thigh and the gluteus maximus muscle in the buttocks.  The
former can be identified by imagining a three-by-three grid on the top
of the thigh; the injection site should be in the outer-middle third.
If injecting into the buttocks, aim to inject into the outer-upper
quadrant.  The thigh is much easier to access for self-administration,
though there is a benefit to cycling through as many injection sites
as possible to avoid scar tissue build-up so if you are comfortable
injecting into the buttocks you should do this too.

Ideally, injection should be done in a warm environment as this will
help to relax the muscle.  Before doing the injection, sterilise any
work surfaces you will be using, thoroughly wash your hands and
assemble all the equipment you will need:

\begin{itemize}
\item The vial of medication
\item Two alcohol swabs, or cotton balls or pads dipped in rubbing
  alcohol for sterialising the vial and the injection site
\item 1 ml syringe
\item 24-48 mm long, 22-30 gauge drawing needle
\item 24-38 mm long, 24-27 gauge injecting needle
\item Sterile cotton pad or swab to catch any bleeding
\item Plaster to apply to the injection site afterwards if needed
\item Contaminated sharps bin (this can be any rigid plastic container
  that is clearly labeled).
\end{itemize}

\textbf{Never reuse a needle or syringe}.  Needle exchange services
are available in many towns and provide injecting supplies for free --
see figure~\ref{fig:needle-exchange-qr} for a page to search for one.
Supplies are also available online and are generally cheap, especially
if bought in bulk.  If you are temporarily unable to access any new
equipment, it is better for you to have a dose late or miss it
entirely than risk infection and injury from reusing equipment.

\begin{figure}
  \centering
  \includegraphics[width=\qrwidth]{needle-exchange-qr}
  \caption{\small Page for finding a needle exchange}
  \label{fig:needle-exchange-qr}
\end{figure}

The procedure for administering the injection is detailed in
section~\ref{subsec:injection-procedure}.  Especially for people new
to self-administering IM injections, it is recommended to carefully
read through the steps to get a clear idea of the procedure
\textbf{before} doing anything.  Two videos, for drawing-up the
medication (filling the syringe) and performing the IM injection are
also linked in figure~\ref{fig:im-injection-videos}.

\begin{figure}
  \centering
  \includegraphics[width=\qrwidth]{drawing-up-video-qr}
  \hspace{2em}
  \includegraphics[width=\qrwidth]{im-injection-video-qr}
  \caption{\small Video guides for self-administering hormone injections}
  \label{fig:im-injection-videos}
\end{figure}

% Procedure should be on a double page spread so people don't have to
% turn a page.
\cleardoublepage

\subsection{IM Injection Procedure}
\label{subsec:injection-procedure}

\begin{enumerate}
\item Remove the protective cap from the vial
\item Sterilise the top of the vial with one of the alcohol swabs
\item Remove the syringe and drawing needle from their packaging
  (including the needle's protective cap) and attach the needle to the
  syringe
\item Set the plunger on the syringe to the volume you will be injecting
\item With the vial upright, insert the needle into the top of the
  vial and fully push down the plunger, inserting the air from the
  syringe into the vial.
\item Keeping the needle in the vial, invert it and position the
  needle so the liquid is fully covering its tip.
\item Slowly pull back on the plunger to fill the syringe until the
  required volume is reached.
\item Check for air bubbles in the syringe. Remove them if there are
  any by flicking the side of the syringe and pressing the plunger in
  again to push them into the vial.  Afterwards, draw the plunger back
  to the required volume.
\item Pull the needle out of the vial, then remove it from the syringe
  and place it in the sharps bin
\item Remove the injecting needle from its packaging and attach it to
  the syringe.
\item Sterilise the injection site with the other alcohol swab and
  wait until the skin is dry.
\item Pull the skin at the injection site in one direction until it is
  lightly taut and the skin is slightly displaced above the muscle.
\item Insert the needle at a 90\textdegree\ angle in one quick and firm
  motion and slowly push the plunger down until the syringe is empty.
  Keep the muscle relaxed.
\item Remove the needle, place both the needle and syringe into the
  sharps bin and release the tension on the skin.
\item Place the cotton pad or swab over the injection site to catch
  any bleeding -- this will often only last for a second or two,
  though don't be alarmed if it lasts for longer.  If bleeding
  continues then place a plaster over the area.
\end{enumerate}

\section{Storage \& Longevity Considerations}

EEn is UV-sensitive, so constant light exposure (especially sunlight)
will degrade it over time.  Vials should therefore be stored somewhere
dark; opaque plastic containers to keep the vials in when not in use
are available for this purpose.  Heat can also shorten the lifespan of
the medication so it should be stored away from any source of heat
like a radiator or cooker.  Additionally, vials should always be
stored upright to avoid the rubber in the stopper perishing from
constant contact with the contents.

\subsection{Vial Coring}

The vials we provide are intended to last for many uses, so there is a
risk of \textit{vial coring} where small pieces of the vial's rubber
stopper break off into the vial.  From there they can be drawn up into
a syringe and injected -- this is not necessarily dangerous but
probably best avoided.  Coring can also lead to a hole forming in the
stopper that is large enough for air, dust and other contaminants to
enter the vial, or for fluid to leak out when the vial is not upright.

In order to reduce these risks, the 45 -- 90\textdegree\ puncture
technique can be used when drawing up: insert the needle at a 45 --
60\textdegree\ angle with the opening of the needle tip facing
upwards, then as the needle is inserted slowly increase the angle to
90\textdegree.

\cleardoublepage

\pagestyle{empty}
\begin{center}
  \null\vfill
  \includegraphics[width=\qrwidth]{repo-qr} \\
  \vspace{1.2em}
  \parbox{0.8\textwidth}{
    \small
    The latest version of this guide, as well as its issue tracker and
    source code are available on the git repository page.
  }
  \vfill
\end{center}

\end{document}
